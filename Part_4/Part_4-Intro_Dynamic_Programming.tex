\documentclass[aspectratio=169]{beamer}
\usetheme{Boadilla}
\usepackage{xcolor}
\usepackage{amsmath}
%\usepackage{parskip}
%%\usecolortheme{seahorse}
\setbeamersize{text margin left=30mm,
			text margin right=30mm}
%Turning-off navigation bar
\beamertemplatenavigationsymbolsempty


%----
%Information to be included in the title page:
%%https://www.overleaf.com/learn/latex/Beamer
\title[Dynamic Programming Part IV] %optional
{Introduction to Dynamic Programming}
\subtitle{Part IV: Application of FE for CRS Utility Function}
\author[PPE Phil(e)]
{PPE Phil(e)\inst{1}}
\institute[] % (optional)
{
  \inst{1}%
  material @ https://github.com/PPEphile
}
\date[2021] % (optional)
{September 2021}
%---

\begin{document}
%---------------------------------------------------------
%Title Page
\frame{\titlepage}
%---------------------------------------------------------


%-------------------------------------------------------
%Handwritten section
{
\setbeamertemplate{background}[grid][step=13]
% change block for exercise/example
\setbeamercolor{block body}{bg=white}
\setbeamercolor{block title}{bg=teal, fg=white}

% for slide colour 
%% Source: https://tex.stackexchange.com/questions/371402/checkered-background-for-beamer-presentation
\setbeamercolor{normal text}{fg=white,bg=black!90}
\setbeamercolor{structure}{fg=white}

\setbeamercolor{alerted text}{fg=red!85!black}

\setbeamercolor{item projected}{use=item,fg=black,bg=item.fg!35}

\setbeamercolor*{palette primary}{use=structure,fg=structure.fg}
\setbeamercolor*{palette secondary}{use=structure,fg=structure.fg!95!black}
\setbeamercolor*{palette tertiary}{use=structure,fg=structure.fg!90!black}
\setbeamercolor*{palette quaternary}{use=structure,fg=structure.fg!95!black,bg=black!80}

\setbeamercolor*{framesubtitle}{fg=white}

\setbeamercolor*{block body}{fg=black,bg=black!10}
\setbeamercolor*{block title alerted}{parent=alerted text,bg=black!15}
%\setbeamercolor*{block title example}{parent=example text,bg=black!15}


\begin{frame}[t]
\begin{block}{Corn-growing with linear utility}
Consider the classical corn growing example with utility $U(c)=c$, $f(k)=2 k$ and $\beta= \frac{1}{3}$ and some $k_0 \geq 0$.
\end{block}
\end{frame}

}



\end{document}
