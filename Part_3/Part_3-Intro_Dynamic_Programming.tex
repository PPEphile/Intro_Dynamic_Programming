\documentclass[aspectratio=169]{beamer}
\usetheme{Boadilla}
\usepackage{xcolor}
\usepackage{amsmath}
%\usepackage{parskip}
%%\usecolortheme{seahorse}
\setbeamersize{text margin left=30mm,
			text margin right=30mm}
%Turning-off navigation bar
\beamertemplatenavigationsymbolsempty

%Unnumbered Footnote
\newcommand\blfootnote[1]{%
\begingroup
\renewcommand\thefootnote{}\footnote{#1}%
\addtocounter{footnote}{-1}%
\endgroup
}

%----
%Information to be included in the title page:
%%https://www.overleaf.com/learn/latex/Beamer
\title[Dynamic Programming Part III] %optional
{Introduction to Dynamic Programming}
\subtitle{Part III: FE for identifying the optimal policy}
\author[PPE Phil(e)]
{PPE Phil(e)\inst{1}}
\institute[] % (optional)
{
  \inst{1}%
  material @ https://github.com/PPEphile
}
\date[2021] % (optional)
{September 2021}
%---

\begin{document}
%---------------------------------------------------------
%Title Page
\frame{\titlepage}
%---------------------------------------------------------

%---------------------------------------------------------
%Table of contents
\begin{frame}
\frametitle{Table of Contents}
\textbf{Identifying the \color{teal} optimal policy} \color{black} 
\begin{itemize}
\item Theorem 4 (necessary condition)
\item Theorem 5 (sufficient condition)
\item \textbf{Example}
\end{itemize}

\blfootnote{Stokey, N.L., Lucas, R.E. and Prescott, E.C. (1989) \textit{Recursive Methods in Economic Dynamics}. Cambridge, Harvard University Press.}

\end{frame}
%---------------------------------------------------------


%Handwritten section
{
\setbeamertemplate{background}[grid][step=13]
% change block for exercise/example
\setbeamercolor{block body}{bg=white}
\setbeamercolor{block title}{bg=teal, fg=white}

\begin{frame}
\frametitle{Least \underline{upper bound}}
	%contents here
\end{frame}

\begin{frame}
\frametitle{\underline{Least} upper bound}
	%contents here
\end{frame}

\begin{frame}[t]
\frametitle{Example}

\end{frame}

}



\end{document}
