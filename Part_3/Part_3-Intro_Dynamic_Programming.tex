\documentclass[aspectratio=169]{beamer}
\usetheme{Boadilla}
\usepackage{xcolor}
\usepackage{amsmath}
\usepackage{undertilde}
%\usepackage{parskip}
%%\usecolortheme{seahorse}
\setbeamersize{text margin left=30mm,
			text margin right=30mm}
%Turning-off navigation bar
\beamertemplatenavigationsymbolsempty

%Unnumbered Footnote
\newcommand\blfootnote[1]{%
\begingroup
\renewcommand\thefootnote{}\footnote{#1}%
\addtocounter{footnote}{-1}%
\endgroup
}

%----
%Information to be included in the title page:
%%https://www.overleaf.com/learn/latex/Beamer
\title[Dynamic Programming Part III] %optional
{Introduction to Dynamic Programming}
\subtitle{Part III: FE for identifying the optimal policy}
\author[PPE Phil(e)]
{PPE Phil(e)\inst{1}}
\institute[] % (optional)
{
  \inst{1}%
  material @ https://github.com/PPEphile
}
\date[2021] % (optional)
{September 2021}
%---

\begin{document}
%---------------------------------------------------------
%Title Page
\frame{\titlepage}
%---------------------------------------------------------

%---------------------------------------------------------
%Table of contents
\begin{frame}
\frametitle{Table of Contents}
\textbf{Identifying the \color{teal} optimal policy} \color{black} 
\begin{itemize}
\item Theorem 4 (necessary condition)
\item Theorem 5 (sufficient condition)
\item \textbf{Example}
\end{itemize}

\blfootnote{Stokey, N.L., Lucas, R.E. and Prescott, E.C. (1989) \textit{Recursive Methods in Economic Dynamics}. Cambridge, Harvard University Press.}

\end{frame}
%---------------------------------------------------------
%
\begin{frame}
\frametitle{Intro}
\begin{itemize}
\item We want to solve: 
\begin{equation}
\max_{0 \leq x_{t+1} \leq f(x_t)} \sum^{\infty}_{t=0} \beta^t F(x_t, x_{t+1}) \tag{SP}
\end{equation}
\item Last time we identified $v^*(x)$, the solution to the (SP)
\item \color{red}\textit{But is that really what we were after?} \color{black}
\end{itemize}

\end{frame}

%Handwritten section
{
\setbeamertemplate{background}[grid][step=13]
% change block for exercise/example
\setbeamercolor{block body}{bg=white}
\setbeamercolor{block title}{bg=teal, fg=white}
\begin{frame}[t]
\begin{block}{Corn-growing with linear utility}
Consider the classical corn growing example with utility $U(c)=c$, $f(k)=2 k$ and $\beta= \frac{1}{3}$ and some $k_0 \geq 0$.
\end{block}
\end{frame}

}



\begin{frame}
\frametitle{A necessary condition}
This gives us an intuitive necessary condition:
\begin{block}{Theorem 4}
If the path $\utilde{x}^{*}$ is optimal, then
\begin{equation*}
v^{*}(x_{t}^{*}) = F(x_{t}^{*}, x_{t+1}^{*}) + \beta v^{*}(x_{t+1}^{*}) = \max_{y \in \Gamma(x)} \left\lbrace F(x, y) + \beta v^{*}(y) \right\rbrace
\end{equation*}
for all $t$.
\end{block}
\end{frame}

\begin{frame}
\frametitle{A sufficient condition}
\begin{block}{Theorem 5}
If the candidate path $\hat{\utilde{x}}$ is feasible and satisfies 
\begin{equation*}
v^{*}(\hat{x}_{t}) = F(\hat{x}_{t}, \hat{x}_{t+1}) + \beta v^{*}(\hat{x}_{t+1})
\end{equation*}
for all $t$ \textbf{and}
\begin{equation*}
\limsup_{t \to \infty} \beta^{t}v^{*}(\hat{x}_{t}) \leq 0 
\end{equation*}
then $\hat{\utilde{x}}$ is optimal.
\end{block}
\end{frame}

%Handwritten section
{
\setbeamertemplate{background}[grid][step=13]
% change block for exercise/example
\setbeamercolor{block body}{bg=white}
\setbeamercolor{block title}{bg=teal, fg=white}

\begin{frame}
\frametitle{Proof}
	%contents here
\end{frame}

\begin{frame}[t]
\frametitle{Example}
\begin{block}{Corn-growing with linear utility}
Consider the classical corn growing example with utility $U(c)=c$, $f(k)=3k$ and $\beta= \frac{1}{3}$ and some $k_0 \geq 0$.
\end{block}
\end{frame}

}



\end{document}
