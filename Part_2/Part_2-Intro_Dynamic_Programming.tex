\documentclass[aspectratio=169]{beamer}
\usetheme{Boadilla}
\usepackage{xcolor}
\usepackage{amsmath}
%\usepackage{parskip}
%%\usecolortheme{seahorse}
\setbeamersize{text margin left=30mm,
			text margin right=30mm}

%Unnumbered Footnote
\newcommand\blfootnote[1]{%
\begingroup
\renewcommand\thefootnote{}\footnote{#1}%
\addtocounter{footnote}{-1}%
\endgroup
}

%----
%Information to be included in the title page:
%%https://www.overleaf.com/learn/latex/Beamer
\title[Dynamic Programming Part II] %optional
{Introduction to Dynamic Programming}
\subtitle{Part II: Functional Equation}
\author[PPE Phil(e)]
{PPE Phil(e)\inst{1}}
\institute[] % (optional)
{
  \inst{1}%
  material @ https://github.com/PPEphile
}
\date[2021] % (optional)
{September 2021}
%---

\begin{document}
%---------------------------------------------------------
%Title Page
\frame{\titlepage}
%---------------------------------------------------------

%---------------------------------------------------------
%Table of contents
\begin{frame}
\frametitle{Table of Contents}
\begin{itemize}
\item \textbf{Verifying \color{magenta} value function}
\begin{itemize}
\item \color{black} Theorem 2 (necessary condition)
\item Theorem 3 (sufficient condition)
\end{itemize}
\item \textbf{Verifying \color{blue}optimal policy}
\begin{itemize}
\item \color{black} Theorem 4 (necessary condition)
\item Theorem 5 (sufficient condition)
\end{itemize}
\end{itemize}

\blfootnote{Stokey, N.L., Lucas, R.E. and Prescott, E.C. (1989) \textit{Recursive Methods in Economic Dynamics}. Cambridge, Harvard University Press.}

\end{frame}
%---------------------------------------------------------

%slide 3
\begin{frame}
\frametitle{The Functional Equation (Recap)}
\begin{itemize}
\item We eliminated infinitely many constraints at the cost of now having to find a function
\item \textbf{Theorem 2}: For any candidate function $v$ to be the true value function $v^*$, it needs to satisfy the (FE).
\end{itemize}
\begin{block}{Functional Equation}
\begin{equation}
v(x) = \max_{y \in \Gamma(x)} \left\lbrace F(x, y) + \beta v(y) \right\rbrace \tag{FE}
\end{equation}
where $\Gamma(x)$ is the set of admissible values of $y$ given the current state $x$.
\end{block}

\end{frame}

% Slide 4
\begin{frame}
\frametitle{Sufficient condition}
\begin{block}{Theorem 3}
If $v$ is real-valued, solves the (FE) \textbf{and}
\begin{equation*}
\lim_{t \to \infty} \beta^{t}v(x_{t}) = 0 \quad \textrm{for all feasible paths of } (x_t)^{\infty}_{t=0}
\end{equation*}
then $v = v^*$
\end{block}
\end{frame}

%Handwritten section
{
\setbeamertemplate{background}[grid][step=13]
%\setbeamercolor{block body example}{fg=structure, bg=white}
%\setbeamertemplate{block example}[shadow=false]
\begin{frame}
\frametitle{Least upper bound}
\begin{itemize}
\item (FE) only necessary but not sufficient
\begin{itemize}
\item e.g. $v(x) = \pm \infty$ is an universal solution
\end{itemize}
\item Supremum: least upper bound
\end{itemize}
\end{frame}

\begin{frame}
	%contents here
\end{frame}

\begin{frame}[t]
\frametitle{Example}
\colorbox{white}{
Consider the classical corn growing example with utility $U(c)=c$, $f(k)=k^{\alpha}$ and $\beta \in [0,1]$.}
\end{frame}

}



\end{document}
